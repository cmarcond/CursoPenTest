\begin{comment}

ATAQUES DE REDE

Observar tecnicas para obter ataque à rede (apesar de escalonamentos e privilégios)

Depois de obter acesso descobrir como usar a rede pra nos beneficiar

Explorar clientes, seridores, switches, roteadores

Foco em criar um nivel de acesso para usar e explorar a organização alvo

Concentrar em protocolos de rede e arquiteturas

----------------------------------------------------------------------------

Obtendo acesso

Mesmo com acesso físico, algumas coisas dificultam o ataque como:
controles de admissão(NAC)
autenticação de porta(IEEE 802.1X)

É necessário manipular um nivel minimo de acesso 

----------------------------------------------------------------------------

Manipulando a Rede

Podemos conseguir acesso a rede generalizada através de manipulção de processos de roteamento

MITM para ajudar a explorar e manipular processos vuneraveis

----------------------------------------------------------------------------

Explorando a rede

Explorar protocolos e sistemas vuneráveis para demonstrar o impacto dos ataques

----------------------------------------------------------------------------

Começando com uma porta

É necessário uma porta de entrada á rede

Pode ser fornecida internamente ou de forma irrestrita através de uma VLAN

Podem existir varios obstaculos para descoberta e avaliação de rede

----------------------------------------------------------------------------

NAC

Sistema de autenticação e validação da rede

Veremos técnicas de como contornalo

----------------------------------------------------------------------------

Primeiro cenário de NAC

Autenticação simples e requesitos minimos de segurança

Não exige que cada cliente tenha o NAC completo, ao invés disso utiliza um navegador
web e usa um agente temporario ActiveX e java para executar verificações de sistema

Cliente se conecta a uma rede inicial restrita(switch)


\end{comment}